\documentclass{beamer}

\usepackage{graphicx}
\usepackage{listings}
%\usepackage{media9}
\usepackage{multimedia}

\usetheme{default}

\usecolortheme[named=white]{structure}
\setbeamercolor{normal text}{fg=white}
\setbeamercolor{background canvas}{bg=black}

\setbeamertemplate{itemize item}{\color{white}$\bullet$}

\setbeamertemplate{frametitle}[default][center]

\setbeamertemplate{navigation symbols}{}

\lstdefinelanguage{Julia}%
  {morekeywords={abstract,break,case,catch,const,continue,do,else,elseif,%
      end,export,false,for,function,immutable,import,importall,if,in,%
      macro,module,otherwise,quote,return,switch,true,try,type,typealias,%
      using,while},%
   basicstyle=\tiny,
   sensitive=true,%
   morecomment=[l]\#,%
   morecomment=[n]{\#=}{=\#},%
   morestring=[s]{"}{"},%
   morestring=[m]{'}{'},%
}[keywords,comments,strings]%

\lstset{%
    language         = Julia,
    %basicstyle       = \ttfamily,
    basicstyle       = \footnotesize\ttfamily,
    keywordstyle     = \bfseries\color{green},
    stringstyle      = \color{magenta},
    commentstyle     = \color{green},
    showstringspaces = false,
}

\begin{document}

\begin{frame}
\center \Huge{I want to talk about an algorithm...}
\end{frame}

\begin{frame}
\begin{columns}

\onslide<1->
\column{0.5\textwidth}
\center Tracking Vortices
\includegraphics[width=0.75\textwidth]{vortex_tracking.png}

\column{0.5\textwidth}

\onslide<2->

\center Terminal Game
\includegraphics[width=\textwidth]{terminal_game.png}

\vspace{0.5cm}

\onslide<3->
\center Mine Sweeper

\includegraphics[width=\textwidth]{minesweep.png}
\end{columns}
\end{frame}


\begin{frame}
\center \Huge Flood Fill
\end{frame}

\begin{frame}
\frametitle{Bucket Fill}
\end{frame}

\begin{frame}
\frametitle{How to Flood Fill}
\center 2 parts to the method

\begin{columns}
\column{0.5\textwidth}
\onslide<2->
\center Finding Domain
\column{0.5\textwidth}
\onslide<3->
\center Filling Domain
\end{columns}
\begin{columns}
\column{0.5\textwidth}
\onslide<4->
\includegraphics[width=\textwidth]{simple_circle.png}
\column{0.5\textwidth}
\onslide<5->
\vspace{-0.5cm}
\center \includegraphics[width=0.85\textwidth]{grid_1.png}
\end{columns}
\onslide<6->
\center depth and breadth first traversal
\end{frame}

\begin{frame}[fragile]
\frametitle{Depth-first Traversal}
\begin{columns}
\column{0.6\textwidth}
\begin{lstlisting}
function DFS_recursive(n::Node)
    println(n.ID)

    for child in n.children
        DFS_recursive(child)
    end
end
\end{lstlisting}
\column{0.4\textwidth}
\movie[autostart]{\includegraphics[width=\textwidth]{tree.png}}{DFS.mp4}
\end{columns}
\end{frame}

\begin{frame}[fragile]
\frametitle{Depth-first Flood Fill}
\center \movie[autostart]{\includegraphics[width=0.75\textwidth]{unfilled.png}}{recurse_animation.mp4}

\end{frame}

\begin{frame}[fragile]
\frametitle{Breadth-first Traversal}
\begin{columns}
\column{0.6\textwidth}
\begin{lstlisting}
function BFS_queue(n::Node)
    q = Queue{Node}()
    enqueue!(q, n)

    while(length(q) > 0)
        println(front(q).ID)
        temp = dequeue!(q)
        for child in temp.children
            enqueue!(q, child)
        end
    end
end
\end{lstlisting}
\column{0.4\textwidth}
\movie[autostart]{\includegraphics[width=\textwidth]{tree.png}}{BFS.mp4}
\end{columns}
\end{frame}

\begin{frame}
\frametitle{Breadth-first Flood Fill}
\center \movie[autostart]{\includegraphics[width=0.75\textwidth]{unfilled.png}}{queue_animation.mp4}
\end{frame}

\begin{frame}
\frametitle{Optimizing Breadth First}
\begin{overprint}
\onslide<1> \center \includegraphics[width=0.75\textwidth]{grid_1.png}
\onslide<2> \center \includegraphics[width=0.75\textwidth]{grid_2.png}
\onslide<3> \center \includegraphics[width=0.75\textwidth]{grid_3.png}
\onslide<4> \center \includegraphics[width=0.75\textwidth]{grid_4.png}
\onslide<5> \center \includegraphics[width=0.75\textwidth]{grid_5.png}
\onslide<6> \center \includegraphics[width=0.75\textwidth]{grid_6.png}
\end{overprint}
\end{frame}

\begin{frame}
\frametitle{Two methods, same result}
\begin{columns}
\column{0.5\textwidth}
\movie[autostart]{\includegraphics[width=\textwidth]{unfilled.png}}{recurse_animation.mp4}
\column{0.5\textwidth}
\movie[autostart]{\includegraphics[width=\textwidth]{unfilled.png}}{queue_animation.mp4}
\end{columns}
\end{frame}

\begin{frame}
\frametitle{Applications}
\center \movie[autostart]{\includegraphics[width=0.75\textwidth]{maze.png}}{maze.mp4}
\end{frame}

\end{document} 
